% -*-latex-*-

% Definitions.
\newcommand{\defn}[1]{{\bfseries\itshape #1}}
\newcommand{\defi}[1]{{\bfseries\itshape #1}}


%% Stuff for proofs.
%\newcommand{\proof}{\noindent{\itshape Proof:}\hspace*{1em}}
%\newcommand{\proofsketch}{\noindent{\itshape Proof Sketch:}\hspace*{1em}}
%\newcommand{\proofnote}[1]{\noindent{\itshape Proof:}\hspace*{.1em}
%  \footnote{#1}\hspace*{0.5em}}
%\newcommand{\qed}{\nolinebreak[1]~~~\hspace*{\fill}
%  \rule{5pt}{5pt}\vspace*{\parskip}\vspace*{1ex}}
%\newcommand{\proofoftheorem}[1]{\noindent{\itshape Proof of Theorem
%  \ref{thm:#1}:}\hspace*{1em}}

% Code listings.
\newcounter{codeLineCntr}
\newcommand{\codeLine}
 {\refstepcounter{codeLineCntr}{\small\thecodeLineCntr}}
\newenvironment{codeListing}
 {\setcounter{codeLineCntr}{0}\small\ttfamily\begin{tabbing}}
 {\end{tabbing}}

% Footnote commands.
\newcommand{\footnotenonumber}[1]{{\def\thempfn{}\footnotetext{#1}}}

% Margin notes - use \notesfalse to turn off notes.
%\setlength{\marginparwidth}{0.6in}
%\reversemarginpar
%\newif\ifnotes
%\notestrue
%\newcommand{\longnote}[1]{
%  \ifnotes
%    {\medskip\noindent Note:\marginpar[\hfill$\Longrightarrow$]
%      {$\Longleftarrow$}{#1}\medskip}
%  \fi}
%\newcommand{\note}[1]{
%  \ifnotes
%    {\marginpar{\raggedright{\tiny #1}}}
%  \fi}

% Stuff not wanted.
\newcommand{\punt}[1]{}
\newcommand{\insuffspace}[1]{}

% Sectioning commands.
\newcommand{\subsec}[1]{\subsection{\boldmath #1 \unboldmath}}
\newcommand{\subheading}[1]{\subsubsection*{#1}}
\newcommand{\subsubheading}[1]{\paragraph*{#1}}

% Reference shorthands.
\newcommand{\chref}[1]{Chapter~\ref{ch:#1}}
\newcommand{\chreftwo}[2]{Chapters \ref{ch:#1} and~\ref{ch:#2}}
%\newcommand{\secref}[1]{Section~\ref{sec:#1}}
%\newcommand{\secreftwo}[2]{Sections \ref{sec:#1} and~\ref{sec:#2}}
\newcommand{\secref}[1]{Sec.~\ref{sec:#1}}
\newcommand{\secreftwo}[2]{Sec. \ref{sec:#1} and~\ref{sec:#2}}
\newcommand{\appref}[1]{Appendix~\ref{app:#1}}
%\newcommand{\figref}[1]{Figure~\ref{fig:#1}}
%\newcommand{\figreftwo}[2]{Figures \ref{fig:#1} and~\ref{fig:#2}}
\newcommand{\figref}[1]{Fig.~\ref{fig:#1}}
\newcommand{\figreftwo}[2]{Figs. \ref{fig:#1} and~\ref{fig:#2}}
\newcommand{\figpageref}[1]{page~\pageref{fig:#1}}
\newcommand{\tblref}[1]{Table~\ref{tbl:#1}}
\newcommand{\tblreftwo}[2]{Tables \ref{tbl:#1} and~\ref{tbl:#2}}
\newcommand{\stref}[1]{Step~\ref{step:#1}}
\newcommand{\caseref}[1]{Case~\ref{case:#1}}
\newcommand{\lineref}[1]{Line~\ref{line:#1}}
\newcommand{\linereftwo}[2]{Lines \ref{line:#1} and~\ref{line:#2}}
\newcommand{\linerefrange}[2]{Lines \ref{line:#1} through~\ref{line:#2}}
\newcommand{\thmref}[1]{Theorem~\ref{thm:#1}}
\newcommand{\thmreftwo}[2]{Theorems \ref{thm:#1} and~\ref{thm:#2}}
%\newcommand{\propref}[1]{Proposition~\ref{prop:#1}}
%\newcommand{\propreftwo}[2]{Propositions \ref{prop:#1} and~\ref{prop:#2}}
\newcommand{\propref}[1]{Prop.~\ref{prop:#1}}
\newcommand{\propreftwo}[2]{Props. \ref{prop:#1} and~\ref{prop:#2}}
\newcommand{\thmpageref}[1]{page~\pageref{thm:#1}}
%\newcommand{\eqnref}[1]{Equation~(\ref{eqn:#1})}
%\newcommand{\eqnreftwo}[2]{Equations~(\ref{eqn:#1}) and~(\ref{eqn:#2})}
%\newcommand{\eqnrefthree}[3]{Equations~(\ref{eqn:#1}),~(\ref{eqn:#2}) and~(\ref{eqn:#3})}
\newcommand{\eqnref}[1]{Eq.~(\ref{eqn:#1})}
\newcommand{\eqnreftwo}[2]{Eqs.~(\ref{eqn:#1}) and~(\ref{eqn:#2})}
\newcommand{\eqnrefthree}[3]{Eqs.~(\ref{eqn:#1}),~(\ref{eqn:#2}) and~(\ref{eqn:#3})}
\newcommand{\lemref}[1]{Lemma~\ref{lem:#1}}
\newcommand{\lemreftwo}[2]{Lemmas \ref{lem:#1} and~\ref{lem:#2}}
\newcommand{\lemrefthree}[3]{Lemmas \ref{lem:#1}, \ref{lem:#2},
and~\ref{lem:#3}}
\newcommand{\lemreffour}[4]{Lemmas \ref{lem:#1}, \ref{lem:#2},
\ref{lem:#3}, and~\ref{lem:#4}}
\newcommand{\lempageref}[1]{page~\pageref{lem:#1}}
\newcommand{\corref}[1]{Corollary~\ref{cor:#1}}
\newcommand{\defref}[1]{Def.~\ref{def:#1}}
\newcommand{\defreftwo}[2]{Def. \ref{def:#1} and~\ref{def:#2}}
\newcommand{\defpageref}[1]{page~\pageref{def:#1}}
\newcommand{\eqpageref}[1]{page~\pageref{eq:#1}}
\newcommand{\ineqref}[1]{Inequality~(\ref{ineq:#1})}
\newcommand{\ineqreftwo}[2]{Inequalities (\ref{ineq:#1}) and~(\ref{ineq:#2})}
\newcommand{\ineqpageref}[1]{page~\pageref{ineq:#1}}

% Useful shorthands.
\newcommand{\argmin}{\arg\min}
\newcommand{\argmax}{\arg\max}
\newcommand{\eps}{\varepsilon}
\newcommand{\set}[1]{\left\{ #1 \right\}}
\newcommand{\abs}[1]{\left| #1\right|}
\newcommand{\card}[1]{\left| #1\right|}
\newcommand{\norm}[1]{\left\| #1\right\|}
\newcommand{\floor}[1]{\left\lfloor #1 \right\rfloor}
\newcommand{\ceil}[1]{\left\lceil #1 \right\rceil}
  \renewcommand{\choose}[2]{{{#1}\atopwithdelims(){#2}}}
\newcommand{\ang}[1]{\langle#1\rangle}
\newcommand{\paren}[1]{\left(#1\right)}
\newcommand{\prob}[1]{\Pr\left\{ #1 \right\}}
\newcommand{\expect}[1]{\mathrm{E}\left[ #1 \right]}
\newcommand{\expectsq}[1]{\mathrm{E}^2\left[ #1 \right]}
\newcommand{\variance}[1]{\mathrm{Var}\left[ #1 \right]}
\newcommand{\twodots}{\mathinner{\ldotp\ldotp}}
\newcommand{\Otilde}[1]{\tilde{O}(#1)}
\newcommand{\myth}{^{\hbox{\scriptsize{th}}}}

% Standard number sets.
\newcommand{\reals}{{\mathrm{I}\!\mathrm{R}}}
\newcommand{\integers}{\mathbf{Z}}
\newcommand{\naturals}{{\mathrm{I}\!\mathrm{N}}}
\newcommand{\rationals}{\mathbf{Q}}
\newcommand{\complex}{\mathbf{C}}

%
\newcommand{\poly}[1]{\id{poly}(#1)}
\newcommand{\polylog}[1]{\id{polylog}(#1)}

% Special styles.
\newcommand{\proc}[1]{\ifmmode\mbox{\textsc{#1}}\else\textsc{#1}\fi}
\newcommand{\procdecl}[1]{
  \proc{#1}\vrule width0pt height0pt depth 7pt \relax}
\newcommand{\id}[1]{\ifmmode\mathit{#1}\else\textit{#1}\fi}
\newcommand{\func}[1]{\ifmmode\mathrm{#1}\else\textrm{#1}\fi}

% Multiple cases.
\renewcommand{\cases}[1]{\left\{ \begin{array}{ll}#1\end{array}\right.}
\newcommand{\cif}[1]{\mbox{if $#1$}}
