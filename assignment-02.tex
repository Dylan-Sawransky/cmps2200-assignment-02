\documentclass[12pt]{article}
\usepackage{fullpage}
\usepackage{amsmath}

\begin{document}
\title{\sf CMPS 2200: Homework 2}
\author{}
\date{\vspace{-.5in}\sf Monday, Sep 28}
\maketitle\thispagestyle{empty}
% -*-latex-*-

% Definitions.
\newcommand{\defn}[1]{{\bfseries\itshape #1}}
\newcommand{\defi}[1]{{\bfseries\itshape #1}}


%% Stuff for proofs.
%\newcommand{\proof}{\noindent{\itshape Proof:}\hspace*{1em}}
%\newcommand{\proofsketch}{\noindent{\itshape Proof Sketch:}\hspace*{1em}}
%\newcommand{\proofnote}[1]{\noindent{\itshape Proof:}\hspace*{.1em}
%  \footnote{#1}\hspace*{0.5em}}
%\newcommand{\qed}{\nolinebreak[1]~~~\hspace*{\fill}
%  \rule{5pt}{5pt}\vspace*{\parskip}\vspace*{1ex}}
%\newcommand{\proofoftheorem}[1]{\noindent{\itshape Proof of Theorem
%  \ref{thm:#1}:}\hspace*{1em}}

% Code listings.
\newcounter{codeLineCntr}
\newcommand{\codeLine}
 {\refstepcounter{codeLineCntr}{\small\thecodeLineCntr}}
\newenvironment{codeListing}
 {\setcounter{codeLineCntr}{0}\small\ttfamily\begin{tabbing}}
 {\end{tabbing}}

% Footnote commands.
\newcommand{\footnotenonumber}[1]{{\def\thempfn{}\footnotetext{#1}}}

% Margin notes - use \notesfalse to turn off notes.
%\setlength{\marginparwidth}{0.6in}
%\reversemarginpar
%\newif\ifnotes
%\notestrue
%\newcommand{\longnote}[1]{
%  \ifnotes
%    {\medskip\noindent Note:\marginpar[\hfill$\Longrightarrow$]
%      {$\Longleftarrow$}{#1}\medskip}
%  \fi}
%\newcommand{\note}[1]{
%  \ifnotes
%    {\marginpar{\raggedright{\tiny #1}}}
%  \fi}

% Stuff not wanted.
\newcommand{\punt}[1]{}
\newcommand{\insuffspace}[1]{}

% Sectioning commands.
\newcommand{\subsec}[1]{\subsection{\boldmath #1 \unboldmath}}
\newcommand{\subheading}[1]{\subsubsection*{#1}}
\newcommand{\subsubheading}[1]{\paragraph*{#1}}

% Reference shorthands.
\newcommand{\chref}[1]{Chapter~\ref{ch:#1}}
\newcommand{\chreftwo}[2]{Chapters \ref{ch:#1} and~\ref{ch:#2}}
%\newcommand{\secref}[1]{Section~\ref{sec:#1}}
%\newcommand{\secreftwo}[2]{Sections \ref{sec:#1} and~\ref{sec:#2}}
\newcommand{\secref}[1]{Sec.~\ref{sec:#1}}
\newcommand{\secreftwo}[2]{Sec. \ref{sec:#1} and~\ref{sec:#2}}
\newcommand{\appref}[1]{Appendix~\ref{app:#1}}
%\newcommand{\figref}[1]{Figure~\ref{fig:#1}}
%\newcommand{\figreftwo}[2]{Figures \ref{fig:#1} and~\ref{fig:#2}}
\newcommand{\figref}[1]{Fig.~\ref{fig:#1}}
\newcommand{\figreftwo}[2]{Figs. \ref{fig:#1} and~\ref{fig:#2}}
\newcommand{\figpageref}[1]{page~\pageref{fig:#1}}
\newcommand{\tblref}[1]{Table~\ref{tbl:#1}}
\newcommand{\tblreftwo}[2]{Tables \ref{tbl:#1} and~\ref{tbl:#2}}
\newcommand{\stref}[1]{Step~\ref{step:#1}}
\newcommand{\caseref}[1]{Case~\ref{case:#1}}
\newcommand{\lineref}[1]{Line~\ref{line:#1}}
\newcommand{\linereftwo}[2]{Lines \ref{line:#1} and~\ref{line:#2}}
\newcommand{\linerefrange}[2]{Lines \ref{line:#1} through~\ref{line:#2}}
\newcommand{\thmref}[1]{Theorem~\ref{thm:#1}}
\newcommand{\thmreftwo}[2]{Theorems \ref{thm:#1} and~\ref{thm:#2}}
%\newcommand{\propref}[1]{Proposition~\ref{prop:#1}}
%\newcommand{\propreftwo}[2]{Propositions \ref{prop:#1} and~\ref{prop:#2}}
\newcommand{\propref}[1]{Prop.~\ref{prop:#1}}
\newcommand{\propreftwo}[2]{Props. \ref{prop:#1} and~\ref{prop:#2}}
\newcommand{\thmpageref}[1]{page~\pageref{thm:#1}}
%\newcommand{\eqnref}[1]{Equation~(\ref{eqn:#1})}
%\newcommand{\eqnreftwo}[2]{Equations~(\ref{eqn:#1}) and~(\ref{eqn:#2})}
%\newcommand{\eqnrefthree}[3]{Equations~(\ref{eqn:#1}),~(\ref{eqn:#2}) and~(\ref{eqn:#3})}
\newcommand{\eqnref}[1]{Eq.~(\ref{eqn:#1})}
\newcommand{\eqnreftwo}[2]{Eqs.~(\ref{eqn:#1}) and~(\ref{eqn:#2})}
\newcommand{\eqnrefthree}[3]{Eqs.~(\ref{eqn:#1}),~(\ref{eqn:#2}) and~(\ref{eqn:#3})}
\newcommand{\lemref}[1]{Lemma~\ref{lem:#1}}
\newcommand{\lemreftwo}[2]{Lemmas \ref{lem:#1} and~\ref{lem:#2}}
\newcommand{\lemrefthree}[3]{Lemmas \ref{lem:#1}, \ref{lem:#2},
and~\ref{lem:#3}}
\newcommand{\lemreffour}[4]{Lemmas \ref{lem:#1}, \ref{lem:#2},
\ref{lem:#3}, and~\ref{lem:#4}}
\newcommand{\lempageref}[1]{page~\pageref{lem:#1}}
\newcommand{\corref}[1]{Corollary~\ref{cor:#1}}
\newcommand{\defref}[1]{Def.~\ref{def:#1}}
\newcommand{\defreftwo}[2]{Def. \ref{def:#1} and~\ref{def:#2}}
\newcommand{\defpageref}[1]{page~\pageref{def:#1}}
\newcommand{\eqpageref}[1]{page~\pageref{eq:#1}}
\newcommand{\ineqref}[1]{Inequality~(\ref{ineq:#1})}
\newcommand{\ineqreftwo}[2]{Inequalities (\ref{ineq:#1}) and~(\ref{ineq:#2})}
\newcommand{\ineqpageref}[1]{page~\pageref{ineq:#1}}

% Useful shorthands.
\newcommand{\argmin}{\arg\min}
\newcommand{\argmax}{\arg\max}
\newcommand{\eps}{\varepsilon}
\newcommand{\set}[1]{\left\{ #1 \right\}}
\newcommand{\abs}[1]{\left| #1\right|}
\newcommand{\card}[1]{\left| #1\right|}
\newcommand{\norm}[1]{\left\| #1\right\|}
\newcommand{\floor}[1]{\left\lfloor #1 \right\rfloor}
\newcommand{\ceil}[1]{\left\lceil #1 \right\rceil}
  \renewcommand{\choose}[2]{{{#1}\atopwithdelims(){#2}}}
\newcommand{\ang}[1]{\langle#1\rangle}
\newcommand{\paren}[1]{\left(#1\right)}
\newcommand{\prob}[1]{\Pr\left\{ #1 \right\}}
\newcommand{\expect}[1]{\mathrm{E}\left[ #1 \right]}
\newcommand{\expectsq}[1]{\mathrm{E}^2\left[ #1 \right]}
\newcommand{\variance}[1]{\mathrm{Var}\left[ #1 \right]}
\newcommand{\twodots}{\mathinner{\ldotp\ldotp}}
\newcommand{\Otilde}[1]{\tilde{O}(#1)}
\newcommand{\myth}{^{\hbox{\scriptsize{th}}}}

% Standard number sets.
\newcommand{\reals}{{\mathrm{I}\!\mathrm{R}}}
\newcommand{\integers}{\mathbf{Z}}
\newcommand{\naturals}{{\mathrm{I}\!\mathrm{N}}}
\newcommand{\rationals}{\mathbf{Q}}
\newcommand{\complex}{\mathbf{C}}

%
\newcommand{\poly}[1]{\id{poly}(#1)}
\newcommand{\polylog}[1]{\id{polylog}(#1)}

% Special styles.
\newcommand{\proc}[1]{\ifmmode\mbox{\textsc{#1}}\else\textsc{#1}\fi}
\newcommand{\procdecl}[1]{
  \proc{#1}\vrule width0pt height0pt depth 7pt \relax}
\newcommand{\id}[1]{\ifmmode\mathit{#1}\else\textit{#1}\fi}
\newcommand{\func}[1]{\ifmmode\mathrm{#1}\else\textrm{#1}\fi}

% Multiple cases.
\renewcommand{\cases}[1]{\left\{ \begin{array}{ll}#1\end{array}\right.}
\newcommand{\cif}[1]{\mbox{if $#1$}}


Complete the problems below and turn in your written answers on Canvas
or in person. I encourage you to
verbally discuss approaches to solving individual problems, but your written
submission must be your work, and only your work.

\begin{enumerate}

\item Suppose that for a given task you are choosing between the following three algorithms:
  
  \begin{itemize}
    \item Algorithm $\mathcal{A}$ solves problems by dividing them into five
      subproblems of half the size, recursively solving each
      subproblem, and then combining the solutions in linear time.
      \item Algorithm $\mathcal{B}$ solves problems of size $n$ by recursively
        solving two subproblems of size $n-1$ and then combining the
        solutions in constant time.
        \item Algorithm $\mathcal{C}$ solves problems of size $n$ by
          dividing them into nine subproblems of size $n/3$,
          recursively solving each subproblem, and then combining the
          solutions in $O(n^2)$ time.
        \end{itemize}

        What are the asymptotic running times of each of these
        algorithms? Which algorithm would you choose?

\punt{  \textbf{Solution:} We write the recursive formula of running time of each algorithm in the following: 

\begin{enumerate}

\item Algorithm $\mathcal{A}$: $T(n)=5T(n/2)+\Theta(n) \implies T(n)= \Theta(n^{\log 5})$.

\item Algorithm $\mathcal{B}$: $T(n)=2T(n-1)+\Theta(1) \implies T(n)=\Theta(2^n)$.

\item Algorithm $\mathcal{C}$: $T(n)=9T(n/3)+O(n^2) \implies T(n)=O(n^2\log n)$.

To compute the problem we definitely must choose the third algorithm with smallest running time.
\end{enumerate}
}

\item For each recurrence below, use either the brick method,
  substitution, or induction to derive an
  asymptotic upper bound.
  \begin{enumerate}
  \item $T(n)=2T(n/3)+1$
  \item $T(n)=5T(n/4)+n$
  \item $T(n)=7T(n/7)+n$
  \item $T(n)=9T(n/3)+n^2$
  \item $T(n)=8T(n/2)+n^3$
    \item $T(n)=49T(n/25)+n^{3/2}\log n$
    \item $T(n)=T(n-1)+2$
    \item $T(n)= T(n-1)+n^c$, with $c\geq 1$
      \item $T(n)=T(\sqrt{n})+1$
      \end{enumerate}

 \punt{

   % need to rewrite using brick method
   
 \textbf{Solution:} Solve the following recurrences to obtain asymptotic\textbf {Problem 2.5.} We only use Master's theorem for case (a) to (f). We only need to substitute the values that we are given into the formula (we can use the theorem and get $\Theta(.)$ since all $f(n)$s are in term of $\Theta(.)$):
\begin{enumerate}
            \item $T(n)=2T(n/3)+1$: Since $a=2, b=3, d=0 (d<\log_3 2) \implies T(n)= \Theta(n^{\log_3 2})$.
           
            \item $T(n)=5T(n/4)+n$: Since $a=5, b=4, d=1 (d<\log_4 5) \implies T(n)= \Theta(n^{\log_4 5})$.
           
            \item $T(n)=7T(n/7)+n$: Since $a=7, b=7, d=1 (d=\log_7 7) \implies T(n)= \Theta(n\log n)$.
           
            \item $T(n)=9T(n/3)+n^2$: Since $a=9, b=3, d=2 (d=\log_3 9) \implies T(n)= \Theta(n^2\log n)$.
   
    \item $T(n)=8T(n/2)+n^3$: Since $a=8, b=2, d=3 (d=\log_2 8) \implies T(n)= \Theta(n^3\log n)$.
   
    \item $T(n)=49T(n/25)+n^{3/2}\log n$: Since $a=49, b=25,~ \forall\varepsilon>0 ~d=3/2+\varepsilon (d> \log_25 49) \implies T(n)= \Theta(n^{3/2}{\log n})$.
   
    \item $T(n)=T(n-1)+2$: \\
    $T(n)=T(n-2)+4$ \\
    $\vdots $\\
    $T(n)=T(1)+2(n-1)$  $\implies T(n)=\Theta (n)$.
    
    \item $T(n)= T(n-1)+n^c (c\geq 1)$:  $T(n) = n^c+(n-1)^c+(n-2)^c+\cdots +2^c+1 < \underbrace{n^c+n^c+n^c+\cdots + n^c+n^c}_\text{n times}= n.n^c= n^{c+1}1$, which is an upper bound when $c'=1 , n_0=1$. To get the same for lower bound:
   
    $\underbrace{(n/2)^c+(n/2)^c+(n/2)^c+\cdots+ (n/2)^c}_\text{n/2 times} < n^c+(n-1)^c+(n-2)^c+\cdots +2^c+1$, when $c''=(\frac{1}{2^c}) , n_0=1$.
   
    Which implies that $T(n)= \Theta(n^{c+1})$.
    
    
    
    \item $T(n)= T(n-1)+ c^n$: $T(n)= T(n-2)+ c^n+c^{n-1} \implies \cdots T(n)= T(0) + \Sigma^n_{i=1} c^i= \Theta (c^n)$
   
    \item  $T(n)=2T(n-1)+1$: $T(n)=2(2T(n-2)+1)+1 \implies \cdots T(n)= 2^kT(n-k)+\sum_{i=0}^{k-1}2^i$. Since $k$ must be $n$, thus: $T(n)= \Theta(2^n)$.
    
    
    
    \item $T(n)=T(\sqrt{n})+1$. Taking square root implies that $n$ must be $2^{2^t}$ for some $t>0$. Let $S(t)= T(2^{2^t})$, then we rewrite the latter recursive formula as:
    $S(t)=S(t-1)+1 \implies S(t)=\Theta(t)$. Since $t=\log\log n$, therefore $T(n)=S(t)= \Theta(\log\log n)$.
\end{enumerate}
}

\item Now that you have some practice solving recurrences, let's work
  on implementing some algorithms. In lecture we discussed a divide and conquer algorithm for
    integer multiplication. This algorithm takes as input two $n$-bit
    strings $x = \langle x_L, x_R\rangle$ and
    $y=\langle y_L, y_R\rangle$ and computes the product $xy$ by using
    the fact that
    $xy = 2^{n/2}x_Ly_L + 2^{n/2}(x_Ly_R+x_Ry_L) + x_Ry_R.$ Write the
    algorithm specification in SPARC. Then, use the stub functions in
    \texttt{main.py} to implement two algorithms for integer
    multiplication: a divide and conquer algorithm that runs in
    quadratic time, and the Karatsaba-Ofman algorithm running in
    subquadratic time. Then test the empirical running times across a
    variety of inputs to test whether your code scales in the manner
    described by the asymptotic runtime?

    \punt{
      \item In lecture we also showed that the running time of this
        divide-and-conquer integer multiplication algorithm is really no
        faster than the grade school multiplication. This is because
        we require 4 recursive calls on inputs of size $n/2$, which
        compute $x_Ly_L, x_Ly_R, x_Ry_L$ and $x_Ry_R.$ Then combining
        solutions requires $O(n)$ time. We can improve
        upon our divide-and-conquer algorithm by reducing the number
        of recursions to compute the same terms. First, observe
        that $(x_L + x_R)(y_L + x_R) = x_Ly_L + x_Ry_L + x_Ly_R +
        x_Ry_R$. Using this fact, come up with an asymptotically
        faster algorithm for integer multiplication. How do the work
        and span of this algorithm differ from the standard one? 

        \item Implement your faster algorithm in Python. In practice
          is this algorithm faster than the standard one? How large
          does the input have to be in order for the improved algorithm
          to be 100x faster?
      
    \end{enumerate}}

\end{enumerate}
\end{document}